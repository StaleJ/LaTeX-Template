\documentclass{article}
\usepackage[utf8]{inputenc}

\linespread{1.25}

\usepackage{fancyhdr}
\usepackage{amsmath}
\usepackage{graphicx}
\usepackage{apacite}
\usepackage{amssymb}
\usepackage{nicefrac}
\usepackage{dirtytalk}
\usepackage{tikz}
\usepackage{IEEEtrantools}

%\usepackage{showframe}
\usepackage{geometry}
\geometry{
  a4paper,
  total={170mm,257mm},
  left=20mm,
  top=20mm,
}

\newcommand{\R}{\mathbb{R}}
\newcommand{\N}{\mathbb{N}}
\newcommand{\Z}{\mathbb{Z}}
\renewcommand{\iff}{\;\leftrightarrow\;}
\newcommand{\floor}[1][x]{\lfloor #1\rfloor}
\newcommand{\ceil}[1][x]{\lceil #1\rceil}
\newcommand{\mat}[1][A]{\text{\textbf{#1}}}
\newcommand{\I}{\mathbb{1}}
\newcommand{\ihat}{\hat{\imath}}
\newcommand{\jhat}{\hat{\jmath}}
\newcommand{\khat}{\hat{k}}
\newcommand*{\QEDA}{\hfill\ensuremath{\blacksquare}}
\newcommand*{\QEDB}{\null\nobreak\hfill\ensuremath{\square}}%    
\newcommand{\NN}{\mathcal{N}}
\newcommand{\vdashn}{\vdash_\NN}
\newcommand{\GG}{\mathcal{G}}
\newcommand{\vdashg}{\vdash_\mathcal{G}}
\newcommand{\lra}{\Leftrightarrow}
\newcommand\ddfrac[2]{\frac{\displaystyle #1}{\displaystyle #2}}
\newcommand{\underA}{\underline{a}}
\newcommand{\underB}{\underline{b}}

\setlength\parindent{0pt}


\pagestyle{fancy}
\date{}

\begin{document}
\lhead{Obligatorisk innlevering 3 i STAT110}
\rhead{Ståle Jacobsen}
\textsc{Oppgave 1.(a)}\\
Les av $\overline{x}$ og $s$ fra Figur 1 for hver av stasjonene 1, 2 og 3. \\ 
\textbf{Svar:} Vi får at de forskjellige $\overline{x}$ og $s$ til å bli:
\begin{align*}
  \overline{x}_1=125 && \overline{x}_2 = 150 && \overline{x}_3=250\\ 
  s_1=250 && s_2=210 && s_3=395
\end{align*}

\textsc{Oppgave 1.(b)}\\
Lag et $95$ konfidensintervall for hver av $\mu_1, \mu_2$ og $\mu_3$ (totalt tre
intervaller).\\
\textbf{Svar:} Fra definisjonen kan vi finne konfidensintervallet med:
\begin{align*}
  \overline{x}\pm z_{\alpha/2}\frac{s}{\sqrt{n}}
\end{align*}
Først $\mu_1$ og vi har $n_1=18,$ $s_1=250,$ $\overline{x}_1=125,$ og $z_{\alpha/2}=1.96$
\begin{align*}
  125\pm 1.96 \frac{250}{\sqrt{18}}=[9.506,240.494]
\end{align*}
Så $\mu_2$ og vi har $n_2=20$, $s_2=210$ og $\overline{x}_2=150$
\begin{align*}
  150\pm 1.96 \frac{210}{\sqrt{20}}=[57.963, 242.037]
\end{align*}
Til slutt $\mu_3$ og vi har $n_3=13$, $s_3=395$ og $\overline{x}_3=250$
\begin{align*}
  250 \pm 1.96 \frac{395}{\sqrt{13}}=[35.276,464.724] 
\end{align*}

\textsc{Oppgave 1.(c)}\\
Hvilke av de tre konfidensintervallene som du konstruerte under punkt b)
inneholder $\mu_{Ref}=354?$\\
\textbf{Svar:} Vi kan se at $\mu_3$ inneholder $\mu_{Ref}=354$\\

Hvis du har lest kapittel 9: gjenkjenner du dette spørsmåøet som en hypotesetest
(hvilke hypotese?)\\

\textsc{Oppgave 2.(a)}\\
Gitt $n$ observasjoner $t_1,\ldots, t_n$ finn
sannsynlighetsmaksimeringsestimatet $\hat{\theta}$ for $\theta.$ Finn en
nummerisk verdi for $\hat{\theta}$ når en har følgende 10 observerte
betjeningstider: $t_i: 1, 1.4, 2.0, 0.5, 0.7, 2.0, 1.3, 1.1, 1.8, 0.2,$ der
tidsenhet er ett minutt.\\
\textbf{Svar:} Vi har
\begin{align*}
  L(\theta)=\prod_{i=1}^{n} f(t_i ; \theta)=\prod_{i=1}^{n}\theta e^{-\theta t_i}=\theta^n e^{-\theta \sum_{i=1}^{n} t_i}
\end{align*}
Slik at
\begin{align*}
  \ln{L}=n \ln{\theta}-\theta \sum_{i=1}^{n}t_i
\end{align*}
og
\begin{align*}
  \frac{d}{d\theta}\ln{L}=\frac{n}{\theta}- \sum_{i=1}^{n} t_i
\end{align*}
dette git at sannsynlighetsmaksimeringsestimatet er gitt ved
\begin{align*}
  \hat{\theta}=n/\sum_{i=1}^{n}t_i=1/\overline{t}
\end{align*}
Vi får nå estimatet til å bli
\begin{align*}
  \hat{\theta} = 10/\sum_{i=1}^{10}t_i =\frac{10}{12}=0.833 
\end{align*}

\textsc{Oppgave 2.(b)}\\
Er $\hat{\theta}$ en forventningsrett estimator for $\theta?$ Begrunn svaret.\\
\textbf{Svar:} $\hat{\theta}$ er ikke en forventningsrett estimator for
$\theta$, siden
\begin{align*}
  E\left(1 / \overline{T} \right)\ne 1 / E\left( \overline{T} \right)
\end{align*}
  

\textsc{Oppgave 3.(a)}\\
Vis at utrykket for log-likelihood funkjsonen er gitt ved
\begin{align*}
  l(\theta) = [5\ln{(2)}-\ln{(3!)}-\ln{(5!)}]+8\ln{\theta}-3\theta.
\end{align*}
Plott $l(\theta)$ for verdier av $\theta\in [0,10]$ (en grov skisse er OK). For
ca hvilken vedi av $\theta$ har $l(\theta)$ sitt maksimum?\\
Vi vet at
\begin{align*}
  P(X=x)=e^{-\lambda}\frac{1}{x!}\lambda^x && L(\lambda; x_1,\ldots, x_n)=\prod_{j=1}^n e^{-\lambda}\frac{1}{x_j!}\lambda^{x_j}
\end{align*}
Vi får nå
\begin{align*}
  L(X=3, Y=5;\theta, 2\theta)=&e^{-\theta}\frac{1}{3!}\theta^3 e^{-2\theta} \frac{1}{5!} (2\theta)^5\\
  \ln{(L)}=l(\theta)=&\ln{\left( e^{-\theta}\frac{1}{3!}\theta^3 e^{-2\theta} \frac{1}{5!} 2^5 \theta^5 \right)}\\
  =& - \theta - \ln{3!} + 3\ln{\theta}-2\theta - \ln{5!}+5\ln{2}+5\ln{\theta}\\
  =& 5 \ln{2}-\ln{3!}-\ln{5!}+ 8\ln{\theta}-3\theta
\end{align*}
Som vi skulle vise
\begin{center}
  \includegraphics[scale=0.75]{lx.png}
\end{center}
Når $\theta=2.667$ har $l(\theta)$ ca sitt maksimum.\\

\textsc{Oppgave 3.(b)}\\
Beregn sannsynlighetsmaksimeringsestimatoren $\hat{\theta}$ for de observerte
verdiene\\ $x=3$ og $y=5.$
\textbf{Svar:} Begynner med å derivere $l$ med hensyn på $\theta$
\begin{align*}
  \frac{d}{d\theta}l(\theta)=&\frac{d}{d\theta}\left( 5 \ln{2}-\ln{3!}-\ln{5!}+8\ln{\theta}-3\theta \right)\\
  =& \frac{d}{d\theta}\left( \ln{\frac{2\theta^8}{45}} - 3\theta \right)\\
  =& \frac{d}{d\theta}\left( ln{\frac{2\theta^8}{45}} \right)-\frac{d}{d\theta}\left( 3\theta \right \\
  =& \frac{8}{\theta}-3
\end{align*}
Så se på når $\frac{d}{d\theta}l(\theta)=0$ og løse for $\theta$
\begin{align*}
  \frac{8}{\theta}-3 = 0\\
  \theta = \frac{8}{3}
\end{align*}
Sannsynlighetsmaksimeringsestimatoren $\hat{\theta}$ for de observerte verdiene
blir $8/3$
\textsc{Oppgave 4.(a)}\\
Vis at disse har forventning og varians:
\begin{align*}
  E(X_1)=\theta, && V(X_1)=\theta(1-\theta)\\
  E(X_2) = 4\theta, && V(X_2)=4\theta(1-4\theta)\\
  E(X_3) = 5\theta, && V(X_3)=5\theta(1-5\theta)
\end{align*}
Du vil bestemme et estimat for sannsynligheten $\theta$ ut fra de tre
observasjonene $X_1, X_2, X_3.$ Du foreslår at man skal bruke en av disse
estimatorene:
\begin{align*}
  \hat{\theta}_1 = \frac{1}{10}(X_1+X_2+X_3), && \hat{\theta}_2=\frac{1}{3}(X_1+\frac{X_2}{4}+ \frac{X_3}{5}).  
\end{align*}
\textbf{Svar:} Alle tre $X_1, X_2, X_3$ er binomisk fordelt med $n=1$ og
henholdsvis $p_1=\theta, p_2=4\theta$ og $p_5=5\theta.$
For en binomisk fordeling med $n=1$ har vi $E(X)=np=p$ og
$V(X)=np(1-np)=p(1-p).$\\
Vi får
\begin{align*}
  E(X_1)=p_1=\theta && V(X_1)=p_1(1-p_1)=\theta(1-\theta)\\
  E(X_2)=p_2=4\theta && V(X_2)=p_2(1-p_2)=4\theta(1-4\theta)\\
  E(X_3)=p_3=5\theta && V(X_3)=p_3(1-p_3)=5\theta(1-5\theta)
\end{align*}
\textsc{Oppgave 4.(b)}\\
Vis at begge estimatorene er forventingsrette. Fin uttrykk for variansene til
$\hat{\theta}_1$ og $\hat{\theta}_2$ uttrykt vet $\theta.$ Avgjør hvilken
estimator som er best når $\theta=0.2.$\\
\textbf{Svar:} Forventingsrett er at estimatoren har forventning $\theta$\\
\begin{minipage}[t]{.5\linewidth}
  \begin{align*}
    E(\hat{\theta}_1)=&E\left(\frac{1}{10}(X_1+X_2+X_3)\right)\\
    =&\frac{1}{10}(E(X_1)+E(X_2)+E(X_3))\\
    =&\frac{1}{10}(\theta+4\theta+5\theta)=\theta\\
  \end{align*} 
\end{minipage}
\begin{minipage}[t]{.5\linewidth}
  \begin{align*}
    E(\hat{\theta}_2)=&E\left(\frac{1}{3}\left(X_1+ \frac{X_2}{4}+\frac{X_3}{5}\right)\right)\\
    =&\frac{1}{3}\left(E(X_1)+\frac{E(X_2)}{4}+\frac{E(X_3)}{5}\right)\\
    =&\frac{1}{3}\left( \theta + \frac{4\theta}{4}+\frac{5\theta}{5} \right)=\theta
  \end{align*}
\end{minipage}
Begge estimatorene er forventingsrette\\
\begin{figure}[Htb]
  \centering
\hspace{0cm}\begin{minipage}[t]{0.1\linewidth}
  \begin{align*}
    V(\hat{\theta}_2) &= V\left( \frac{1}{10^2}(X_1+X_2+X_3) \right)\\
                      &= \frac{1}{10^2}(V(X_1)+V(X_2)+V(X_3))\\
                      &= \frac{1}{10^2} (\theta(1-\theta)+4\theta(1-4\theta)+5\theta(1-5\theta))\\
                      &=\frac{1}{10^2}(\theta-\theta^2+4\theta-16\theta^2+5\theta-25\theta^2)\\
                      &= \frac{1}{10^2} (10\theta-42\theta^2)\\
                      &= \frac{\theta}{10}-\frac{21\theta^2}{50}
  \end{align*}
\end{minipage}
\begin{minipage}[t]{0.1\linewidth}
  \begin{align*}
    V(\hat{\theta}_2) &= V\left( \frac{1}{3^2} \left( X_1 + \frac{X_2}{4} + \frac{X_3}{5} \right) \right)\\
                      &= \frac{1}{3^2} \left(  V(X_1) + \frac{V(X_2)}{4} + \frac{V(X_3)}{5} \right)\\
                      &= \frac{1}{3^2} \left( \theta(1-\theta) + \frac{4\theta(1-4\theta)}{4} + \frac{5\theta(1-5\theta)}{5} \right)\\
                      &= \frac{1}{3^2} (\theta - \theta^2 + \theta - 4\theta^2+\theta-5\theta^2)\\
                      &= \frac{1}{3^2}(3\theta - 10\theta^2)\\
                      &= \frac{\theta}{3}-\frac{10\theta^2}{3}
  \end{align*}
\end{minipage}
\end{figure}

\newpage
Velger den med minst varians
\begin{align*}
  V(\hat{\theta}_1)=\frac{0.2}{10}-\frac{21\cdot 0.2^2}{50}=0.0032 && V(\hat{\theta}_2)=\frac{0.2}{3}-\frac{10\cdot 0.2^2}{3^2}\approx 0.022
\end{align*}
Derfor er $\hat{\theta}_1$ best\\

\\
\textsc{Oppgave 5.(a)}\\
Hvis en velger et kålhode tilfeldig, hva er sannsynligheten for at dette skal:
i) veie mindre enn 1.5 kg? ii) veie mellom 2 og 2.5 kg? Hva er sannsynligheten
for at vektforskjellen mellom to tilfelidig valgte kålhoder skal være mer enn 1
kg?\\
\textbf{svar:} i) vil finne $P(X<1.5)$
\begin{align*}
  P(X<1.5)=P\left(Z<\frac{1.5-2.2}{0.8}\right)=P(Z<0.875)=0.1908
\end{align*}
ii) Vil finne $P(2<X<2.5)$
\begin{align*}
  P(2<X<2.5) &= P\left(\frac{2-2.2}{0.8}<Z< \frac{2.5-2.2}{0.8} \right)\\
             &=P(-0.25 < Z < 0.375)\\
             &= 0.6443-0.4013=0.24
\end{align*}
Vi har to uavhengige variabler $X_1$ og $X_2$ og vi vil finne $P(|X_1-X_2|>1)$.
\begin{align*}
  \sigma^2= \sqrt{V(X_1) + V(X_2)}=\sqrt{0.64+0.64}=1.13 && \mu=E(X_1) - E(X_2)=2.2-2.2=0
\end{align*}
Som gir
\begin{align*}
  P(|X_1-X_2|>1)=P\left( \frac{|X_1-X_2|}{1.13}> \frac{1}{1.13} \right)= 2P\left( Z< \frac{-1}{1.13} \right)=0.3788
\end{align*}

\textsc{Oppgave 5.(b)}\\
Gitt at et kålhode oppgyller kravet (veier minst 1.5 kg), hva er sannsynligheten
for at det veier mellom 2 og 2.5 kg?\\
\textbf{Svar:}
\begin{align*}
  P(2<X<2.5 | Z > 1.5) &= \frac{P(X<2.5)-P(X<2)}{P(X>1.5)}\\
                       &= P\left( Z < \frac{2.5-\mu}{\sigma} \right)-P\left(Z < \frac{2-\mu}{\sigma} \right)\\
                       &= \frac{0.24}{1-0.1908}=0.297
\end{align*}

\textsc{Oppgave 5.(c)}\\
Hvilke estimator ville di bruke for å estimere $\mu_Y?$ Skriv opp estimatoren og
regn ut estimatet fra dataene i tabellen. Finn et 90\%-konfidensitervall for
$\mu_Y.$ Hva blir lengden av konfidensintervallet? Hvor mange planter måtte vi
minst hatt for at lengden av konfidensintervallet skullet blitt mindre enn 0.2
kg?\\
\textbf{Svar:} En estimator for normalfordeling kan være
$\hat{\mu}_Y=\overline{Y}$ og da blir estimatet
\begin{align*}
  \hat{\mu}_Y=\overline{y}=\frac{1}{10}\sum_{i=1}^{10}y_i=2.65
\end{align*}
90\% konfidensintervallet blir med $Z_{0.05}=1.645$
\begin{align*}
  2.65\pm 1.645 \frac{0.8}{\sqrt{10}}=[2.23, 3.07]
\end{align*}
Lengden av konfidensintervallet blir
\begin{align*}
  2\cdot 1.645 \frac{0.8}{\sqrt{10}}=0.83
\end{align*}
For at lengden $n$ av konfidensintervallet skal bli mindre enn 0.2 kg
\begin{align*}
  2\cdot 1.645\frac{0.8}{\sqrt{n}} &< 0.2\\
                                 n &> \left( \frac{2\cdot 1.645\cdot 0.8}{0.2} \right)^2=13.16
\end{align*}
Derfor må $n=14$ for at bredden av intervallet skal bli mindre enn 0.2\\

\textsc{Oppgave 6.(a)}\\
Finn et 95\% konfidensintervall for $\mu$ basert på dataene som er gitt ovenfor.
Hvordan tolker du en dekningsgrad på 95\%
\\\textbf{Svar:} Vi har at $\overline{x}= 284$ og $s=\sqrt{2342/4}=24.2$.\\
95\% konfidensintervall er gitt ved
\begin{align*}
  284\pm 1.96 \cdot \frac{24.2}{\sqrt{5}}= [262.79, 305.21]
\end{align*}
Det vil si at hvis vi gjentar målingene $n$ ganger vil [nedre, øvre] inneholde
den sanne graden av forurensingen 95\% av gangene.


\end{document}
